% \documentclass{article}
% \begin{document}

Recently, \cite{NPV16,LNPV17} have also introduced reversibility in the
context of the functional and concurrent programming language Erlang
\cite{AVW96}. In particular, a modular semantics for (a subset of)
\emph{Core Erlang} \cite{CGJLNPV04} is first introduced in
\cite{NPV16}, which is particularly appropiate to define a reversible
extension. A reversible, \emph{uncontrolled} (according to the
terminology in~\cite{LMT14}) semantics is then defined in
\cite{LNPV17}. By defining an appropriate notion of concurrent
transitions, the causal consistency of the reversible semantics is
proved. Finally, adding control to the reversible semantics in the
form of a \emph{rollback operator} allows the design of a method to
undo the actions of a given process up to a given
checkpoint---introduced by the programmer. In order to ensure
causal consistency, the rollback action might be propagated to other,
dependent processes.
%
In contrast to previous approaches to reversibility in
$\mu$Oz~\cite{LLMS12,GLM14}, a main difference is that $\mu$Oz is not
distributed: messages move atomically from sender to a chosen queue,
and from the queue to the receiver. Each of the two actions is
performed by a specific thread, hence the action is naturally part of
the associated thread history. In our case, message delivery is not
directly performed by a thread, and only potentially observed when the
target thread performs the receive action (but not necessarily
observed, e.g., if the message does not match the patterns in the
receive). The definition of the notions of causality and concurrency
in this setting is as a consequence much more tricky than in
$\mu$Oz. This difficulty carries over to the definition of the history
information that needs to be tracked, and to how this information is
exploited in the reversible semantics.

%\end{document}


%%%%%%%%%%%%%%%
% Bib items
%%%%%%%%%%%%%%%

@book{AVW96,
  author = {Joe Armstrong and Robert Virding and Claes Wikstr{\"{o}}m and Mike Williams},
  title     = "{Concurrent programming in Erlang (2nd edition)}",
  publisher = {Prentice Hall},
  year      = {1996},
}

@misc{CGJLNPV04,
  author = {Richard Carlsson and Bj{\"{o}}rn Gustavsson and Erik Johansson and Thomas Lindgren and Sven-Olof Nystr\"om and Mikael Pettersson and Robert Virding},
  title = "Core Erlang 1.0.3. Language specification",
  note = "Available from \verb$https://www.it.uu.se/research/$ \verb$group/hipe/cerl/doc/core_erlang-1.0.3.pdf$",
  year = "2004",
}

@inproceedings{GLM14,
  author = {Elena Giachino and Ivan Lanese and Claudio Antares Mezzina},
  title = {Causal-Consistent Reversible Debugging},
  booktitle = {Proc.\ of the 17th International Conference on Fundamental Approaches to Software Engineering ({FASE} 2014)},
  pages = {370--384},
  year = {2014},
  editor = {Stefania Gnesi and Arend Rensink},
  series = {Lecture Notes in Computer Science},
  volume = {8411},
  publisher = {Springer},
}

@inproceedings{LLMS12,
  author = {Michael Lienhardt and Ivan Lanese and Claudio Antares Mezzina and Jean{-}Bernard Stefani},
  title = {A Reversible Abstract Machine and Its Space Overhead},
  booktitle = {Proceedings of the Joint 14th {IFIP} {WG} Int'l Conf.\ on Formal Techniques for Distributed Systems ({FMOODS} 2012) and the 32nd {IFIP} {WG} 6.1 International Conference ({FORTE} 2012)},
  pages = {1--17},
  year = {2012},
  editor = {Holger Giese and Grigore Rosu},
  series = {Lecture Notes in Computer Science},
  volume = {7273},
  publisher = {Springer},
}

@article{LMT14,
  author = {Ivan Lanese and Claudio Antares Mezzina and Francesco Tiezzi},
  title = {Causal-Consistent Reversibility},
  journal = {Bulletin of the {EATCS}},
  volume = {114},
  year = {2014},
}

@misc{LNPV17,
  author = {Ivan Lanese and Naoki Nishida and Adri{\'{a}}n Palacios and Germ{\'{a}}n Vidal},
  title = {A Theory of Reversibility for Erlang},
  year = 2017,
  note = "Submitted for publication",
}

@inproceedings{NPV16,
  author = {Naoki Nishida and Adri{\'{a}}n Palacios and Germ{\'{a}}n Vidal},
  title = {A Reversible Semantics for Erlang},
  booktitle = {Proc.\ of the 26th International Symposium on Logic-Based Program Synthesis and Transformation, LOPSTR 2016},
  year = 2017,
  editor = {Manuel Hermenegildo and Pedro L\'opez-Garc\'{\i}a},
  series = {Lecture Notes in Computer Science},
  publisher = {Springer},
  note = "To appear",
}


