%\documentclass{article}
%
%\begin{document}

Initial ideas for the design and implementation of reversible
object-oriented language have been presented, based on extending Janus
with object-oriented concepts such as classes that encapsulate
behavior and state, inheritance, virtual dispatching, as well as
constructors.  Schultz and Axelsen showed that virtual dispatching can
be seen as reversible decision mechanism that is easily translatable
to a standard reversible programming model such as Janus, and argued
that reversible management of state can be accomplished using
reversible constructors~\cite{joule:2016}. These concepts were
informally described and implemented by source-to-source translation
from the reversible object-oriented language Joule to Janus. A similar
design was adopted for the ROOP reversible object-oriented language,
but fully formalized and implemented by compiling to the Pendulum
reversible ISA~\cite{haulund:2016,roop:2017}. ROOP demonstrates how to
generate low-level code implementing reversible vtable-based
dispatching, and investigates the restrictions that must be imposed on
reversible object-oriented programs to avoid run-time aliasing checks
as found in Joule.

%\bibliographystyle{plain}
%\bibliography{cost-ic1405-wg2-oo-bib}
%
%\end{document}
